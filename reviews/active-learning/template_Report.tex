\documentclass[]{article}


% Title Page
\title{Review of "Active Learning Literature survey"}
\author{Srikanth Muralidharan}


\begin{document}
\maketitle

This paper provides a broad introduction to the popular active learning techniques and principles that could be 
found in the literature. The paper starts with a couple of motivational examples for active learning, follows it up
with the set of scenarios in which learner could queries, and querying strategies used in active learning. Finally, it presents
with some emperical and theoretical analysis of active learning methods.
\par

The contributions of this paper include useful insights into several of the active learning concepts. The analysis
they present include:

\begin{itemize}
	\item {\bf Description and comparison of different query scenarios} The authors provide a clear description of popular
		querying scenarios, with the key assumptions involved in each of the settings, and some popular examples that employed those 
		methods.
	\item {\bf Description of different query strategy frameworks} The authors provide succinct details of commonly used query strategies.
		The authors first classify the query strategies into distinct non-overlapping categories. They also provide details and comparisons
		about each of the methods within each category.
	\item {\bf Emperical and theoretical analysis of active learning} The authors provide some key insights and theoretical analysis to highlight
		important properties of active learning methods, backed by examples from seminal work done in active learning and some of the established
		theoretical concepts in machine learning.
\end{itemize}
\par

In summary, the work provides significant contribution to computer vision research community, with large-scale diverse
clean dataset, with unprecedented set of object categories related by dense hierarchies. The authors clearly demonstrate
the effectiveness of having such a dataset through appropriate applications. The comparisons with previous related work
are very pertinent too.

\end{document}
